\textbf{Sketch methods - to refine}

Methodologically, our modelling workflow is based on the PyPSA (Python for Power System Analysis)\footnote{PyPSA is available at \url{https://github.com/PyPSA/PyPSA}}—an open-source software toolbox for simulating and optimising modern energy systems at high resolution maintained by TU Berlin \cite{Brown-PyPSA,HORSCH-PyPSAEUR}. 
The model developed for this study optimises investment and operational decisions to meet projected electricity demand for the consumers performing clean energy procurement, as well as the demand for other consumers in the European electricity system, while meeting all relevant engineering, reliability, and policy constraints. 
This project is also developed as open-source.\footnote{247-CFE code is available at  \url{https://github.com/PyPSA/247-cfe}}

In this study, we implement a set of new constraints to model a situation when a fraction of corporate and industry (C\&I) demand commit to the 24/7-CFE goal. 
These constraints ensure that a portfolio of carbon-free generation located in the same market as the participating C\&I consumers covers their electricity demand profiles every hour of every day. 
The mathematical implementation allows for examining a scenario space when different shares of demand pursue the 24/7 approach with varying thresholds for meeting the hourly clean energy requirement (e.g. 90\%, 95\%, 100\%) using the definitions provided in Google's paper on methodologies and metrics \cite{Google-methods}.

For comparison, we also model a benchmark scenario where the C\&I consumers meet their annual energy demand on a volumetric basis with renewable energy procurement (the 100\% annual matching), as well as the scenario with no voluntary procurement of carbon-free energy.

Finally, we model scenarios where C\&I consumers are located in different national markets across Europe. 
Thus, we aim to generalise the impacts of 24/7-CFE procurement, accounting for the different patterns of electricity demand, weather, availability of renewable resources, existing electricity generation capacity, and national climate- and energy policies, among other factors.

\subsection{Scenario space}
\label{subsec:scenarios}
text