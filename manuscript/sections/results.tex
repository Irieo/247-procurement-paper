In this section, we present modelling results and related analysis. 
In the analysis, we explore how the 247-CFE and the 100\% annual renewable matching procurement approaches affect both the participating electricity buyers and the electricity system as a whole. 
Focusing on the electricity buyers' perspective, we analyse procurement strategies (i.e. the optimal mix of locally procured capacities of energy generation and storage assets, and electricity imports from the grid), the average costs of procured electricity and carbon emissions associated with the consumers' electricity usage.
When our focus switches to the system perspective, we describe the impact of the \gls{cfe} procurement approaches on \co emissions on the local grids, as well as the impact on back up flexibility needs.

We aim to generalize our findings on 24/7 procurement void individual assumptions by incorporating a large scenario space into the study design.
For convenience, we start with a \textit{base scenario} presented in \cref{subsec:base}.
Afterwards, we explore the scenario space further by presenting selected results that answer the following questions:
\begin{enumerate}[(i)]
\item What if clean electricity buyers have an access to a wider palette of carbon-free technologies that are expected to be available for a commercial scale up in the near future (\cref{subsec:palette})?
\item What if electricity buyers are located in parts of the European electricity grid that are much different to the base scenario? (\cref{subsec:location})?
\item What if electricity procurement takes place in 2030 instead of 2025 (\cref{subsec:time})? 
\item Finally, in \cref{subsec:mechanisms}, we explore how exactly 24/7 CFE commitments help to decarbonize local grids and reveal the individual mechanisms behind this effect that are present across the scenario space.
\end{enumerate}

In the \nameref{sec:discussion}, we summarise our findings for the four questions above and have a look beyond the focus of the model scenarios discussed in the main part of the paper by integrating observations from sensitivity analyses relating to the participation rates of electricity buyers and their diverse consumption profiles, providing an overarching look at the 24/7 procurement and impacts on the electrcity systems.

Given the large scenario space, the manuscript presents results for selected scenario branches that generalise system impacts of 24/7 hourly clean electricity procurement.
For interested reader, we provide results for the entire scenario space in a dashboard that is published online alongside this study, see \nameref{sec:code}.

\subsection{Base scenario}
\label{subsec:base}

\vspace{10pt}
\begin{res}
    24/7 carbon-free electricity (CFE) procurement can eliminate carbon emissiosns of a buyer's electricity consumption and helps to decarbonise the whole electricity system.
\end{res}

\vspace{5pt}
\begin{res}
    90\%--95\% 24/7 CFE targets have only a small cost premium over 100\% annual renewable matching. 100\% 24/7 CFE can be achieved with solar, wind and battery storage, but at a high cost premium.
\end{res}


The base scenario reflects a case when \textbf{10\%} (on volumetric basis) of \gls{ci} consumers in \textbf{Ireland} in \textbf{2025} voluntrary commits to 24/7 clean electricity procurement goal.
This assumption yields 220~MW of \textbf{base load} that has to be matched with clean electricity. 
The participating consumers can achieve the \gls{cfe} procurement goal by co-optimizing local procurement with of mature technologies (designated as \textbf{palette~1}), i.e., technologies available on a market for commercial use as of now, such as onshore wind, utility-scale solar PV, and battery storage, as well as by importing electricity from the local grid.

%some parts of this paragraph may go to methods
\cref{fig:10-2025-IE-p1-used} shows the fraction of the fraction of hourly demand met with carbon-free electricity depending on the procurement policy that \gls{ci} consumers follow.
We contrast two main cases, one where participating consumers commit to 100\% annual renewable matching policy (100\% RES) and another case where the consumers commit to 24/7 hourly matching policy with various \gls{cfe} scores in a range from 80\% to 100\%. 
A \textit{reference case} depicts a case when the consumers cover their load purely with grid purchases without any policy regarding the origin of electricity.

\begin{figure}
    \centering
    \includegraphics[width=0.95\columnwidth]{plots/10/2025/IE/p1/used.pdf}
    \caption{Fraction of hourly demand met with carbon-free electricity.}
    \label{fig:10-2025-IE-p1-used}
\end{figure}

In the reference case, where \gls{ci} consumers do not procure any resources, relying on grid purchases, only 54\% of demand is met with clean electricity.
100\% RES---the best case for the annual renewable matching policy---results in 83\% fraction. 
Consequently, 24/7~CFE targets beyond 85\% yield higher share of hourly demand met with clean electricity than 100\% annual matching policy.

Another notable trend in \cref{fig:10-2025-IE-p1-used} is that \gls{ci} participants rely more on procured resources and less on grid imports with higher CFE target.
When CFE target approaches 100\%, \gls{ci} consumers cannot rely on electricity imports from the grid, since the grid supply mix contains some carbon content in almost all hours.

\cref{fig:10-2025-IE-p1-ci_emisrate} shows how choice of a procurement policy affects the average emissions rate of \gls{ci} consumers.
Already in 2025, Ireland has a moderately clean electricity system: the emission rate is at 177~g\co/kWh in the reference case. 
If \gls{ci} consumers follow 100\% annual matching policy and procure enough renewable energy, the emission rate is reduced to 60~g\co/kWh.
This yields a considerable emission rate reduction compared to the reference case.
The 24/7 procurement policy with CFE targets above 85\% helps the participating consumers to reduce emissions further beyond the treshhold set by the 100\% annual matching with renewable energy. 
At the 100\% CFE target, participating consumers attain a zero emission rate associated with their electricity consumption, as they perfectly align demand and clean generation on an hourly basis.

\begin{figure}
    \centering
    \includegraphics[width=0.95\columnwidth]{plots/10/2025/IE/p1/ci_emisrate.pdf}
    \caption{Average emissions rate of participating consumers.}
    \label{fig:10-2025-IE-p1-ci_emisrate}
\end{figure}

Voluntary procurement of clean electricity also helps decarbonise electricity system as a whole.
We explore this effect by plotting \co emissions in the local region of the participating consumers, i.e. Ireland.
Without voluntary procurement, the model estimates Irish power sector carbon emissions to be at the level of 4.2~Mt\co -- see a reference case in \cref{fig:10-2025-IE-p1-zone_emissions}.\footnote{For comparison, \href{https://www.seai.ie/data-and-insights/seai-statistics/key-publications/co2-emissions-report/}{seai.ie} reports this value to be at 10.3~Mt\co in 2021. A strong decreasing trend is expected, since Irish government has set ambitious targets to achieve a low-carbon energy system \cite{SEAI}.}
The modelling results show that if only 10\% of \gls{ci} electricity demand commits to 100\% 24/7 carbon-free electricity in 2025, it would reduce Irish emissions by 0.6~Mt\co per year compared to the scenario with no voluntary procurement.
This reduction is equivalent to 14\% of Irish power sector emissions.
100\% annual renewable matching policy can also deliver system-level \co emissions reductions (left bar in \cref{fig:10-2025-IE-p1-zone_emissions}); however, beyond 85\% CFE target, 24/7 hourly matching achieves greater emissions reductions than 100\% annual mathing.

\begin{figure}
    \centering
    \includegraphics[width=0.95\columnwidth]{plots/10/2025/IE/p1/zone_emissions.pdf}
    \caption{\co emissions in the local region (Ireland).}
    \label{fig:10-2025-IE-p1-zone_emissions}
\end{figure}

The following analysis reveals the modelled cost-optimal procurement strategies for each \gls{cfe} policy.
\cref{fig:10-2025-IE-p1-ci_capacity} shows that for 10\% of \gls{ci} load in Ireland (220 MW), the 100\% annual matching policy is met by onshore wind and solar PV generators with combined capacity of 1.3~GW.
Matching demand with clean generation on hourly basis requires a much bigger portfolio of wind and solar than the annual matching policy, up to 4.0~GW of combined capacity to reach the 100\%~CFE.
Batteries are included in the cost-optimal portfolio mix when CFE targets are above 85\%.
Note that for 80\% CFE target, 24/7 participating consumers procure less capacity than for 100\% annual matching policy, as they rely more on grid imports.

The breakdown of costs associated with a procurement policy that participating consumers choose is shown in \cref{fig:10-2025-IE-p1-ci_costandrev}.
Note that revenues from selling the excess electricity to the regional grid at market prices can be treated as ”negative costs” and subtracted from the net procurement cost.
The results show that a CFE targets of 80\%--95\% can be achieved at a small cost premium to 100\% annual renewable matching with solar, wind and batteries.
However, what stands out in the plot is the rapid increase of procurement costs for high 24/7 CFE targets. 
For example, 98\% CFE target has cost premium of 54\% over 100\% the RES policy; while the last 2\% of hourly CFE matching more than doubles the costs.

\begin{figure}
    \centering
    \begin{subfigure}[t]{0.95\columnwidth}
        \centering
        \caption{Portfolio capacity procured by participating consumers.}
        \includegraphics[width=0.95\columnwidth]{plots/10/2025/IE/p1/ci_capacity.pdf}
        \label{fig:10-2025-IE-p1-ci_capacity}
    \end{subfigure}
    \begin{subfigure}[t]{0.95\columnwidth}
        \centering
        \vspace{-0.5cm}
        \caption{Costs of a procurement policy.}
        \includegraphics[width=0.95\columnwidth]{plots/10/2025/IE/p1/ci_costandrev.pdf}
        \label{fig:10-2025-IE-p1-ci_costandrev}
    \end{subfigure}
    \caption{Procurement of clean resources and energy storage by \gls{ci} consumers participating in voluntary clean electricity procurement. 
    \cref{fig:10-2025-IE-p1-ci_capacity} shows power capacity portfolio for each procurement policy.
    \cref{fig:10-2025-IE-p1-ci_costandrev} shows the breakdown of costs associated with a procurement.}
    \label{fig:10-2025-IE-p1-ci_procurement}
\end{figure}


\subsection{Technology access}
\label{subsec:palette}

\vspace{10pt}
\begin{res}
    24/7 CFE procurement triggers investment in new technologies the system will need later: long-duration storage and advanced clean firm generators.
\end{res}

\vspace*{5pt}
\begin{res}
    High targets for 24/7 CFE procurement could have a much smaller cost premium if long duration energy storage or advanced clean firm technologies are available for a commercial scale up.
\end{res}

The cost premium of 24/7 hourly \gls{cfe} procurement presented in \cref{subsec:base} is driven by the variability of renewable power supply.
Indeed, in the periods when not much wind or solar is available, matching every kWh of electricity consumption with carbon-free electricity on hourly basis is not an easy task.
Battery storage technology is cost-optimal for shifting surplus a power supply by a couple of hours.
However, bridging extended periods of low wind and sun with battery storage is expensive.
Also, the results above show that 24/7 participating consumers rely little on grid supply to reach CFE targets of 98\%--100\%, since a local grid's electricity mix almost always includes some fossil-fueled generation.

The results in \cref{fig:10-2025-IE-p23-ci_procurement} reflect a case when 24/7~CFE participants have an access to a wider palette of technologies, including \gls{ldes} and advanced clean firm generators, such as advanced geothermal systems, advanced nuclear or Allam cycle generators with \gls{ccs}.

\cref{fig:10-2025-IE-p2-ci_costandrev} (left panel) shows the breakdown of costs for a case when \gls{ci} consumers have an access to \textbf{palette~2} technologies, i.e., \gls{ldes} complements onshore wind, solar PV and battery storage.
A \gls{ldes} system\footnote{Here we assume 2.5~\euro/kWh underground hydrogen storage in salt caverns.} helps to bridge hours with no renewable feed-in. 
As a result, the required portfolio of renewable capacity and procurement costs are reduced, especially for high CFE targets.
Thus, 100\% 24/7 hourly matching has a cost premium of 42\% above annual matching policy, which is significantly lower than w/o \gls{ldes}.

\cref{fig:10-2025-IE-p3-ci_costandrev} (right panel) illustrates \textbf{palette~3} scenario, i.e., 24/7 participants have an access to all palette~2 technologies, as well as to clean firm generation technologies.
Allam cycle generator with \gls{ccs} is now part of the optimal procurement strategy.
As CFE targets increase, the cost-optimal share of clean firm technology increases.
The clean firm technology helps to fill in the gaps in variable renewable generation to match demand and clean generation on an hourly basis.
The cost premium for 24/7 CFE is further reduced in this case: the 100\% 24/7 CFE costs only 15\% over 100\% annual renewable matching.

\begin{figure*}
    \centering
    \begin{subfigure}{0.5\textwidth}
        \centering
        \caption{Costs of a procurement policy with LDES.}
        \includegraphics[width=0.95\columnwidth]{plots/10/2025/IE/p2/ci_costandrev.pdf}
        \label{fig:10-2025-IE-p2-ci_costandrev}
    \end{subfigure}%
    \begin{subfigure}{0.5\textwidth}
        \centering
        \caption{Costs of a procurement policy with LDES and clean firm generators.}
        \includegraphics[width=0.95\columnwidth]{plots/10/2025/IE/p3/ci_costandrev.pdf}
        \label{fig:10-2025-IE-p3-ci_costandrev}
    \end{subfigure}
    \caption{The breakdown of costs per procurement policy if \gls{ci} consumers located in Ireland have an access to a wider palette of technologies:
    w/ \gls{ldes} (\cref{fig:10-2025-IE-p2-ci_costandrev});
    w/ \gls{ldes} and clean firm generation technologies (\cref{fig:10-2025-IE-p3-ci_costandrev}).
    }
    \label{fig:10-2025-IE-p23-ci_procurement}
\end{figure*}


\subsection{Diverse grid locations}
\label{subsec:location}

\vspace{10pt}
\begin{res}
    Each region of the European electricity system has unique characteristics, such as renewable resources, national policies, degree of interconnection, etc.
    Despite these differences, the model results for 24/7 clean energy procurement show similar trends. 
\end{res}

While we have focused on base scenario reflecting the case of Ireland, the same modelling exercise can be repeated for other regions of the European electricity grid.
Since each region has a unique set of characteristics that depend on local resources, renewable potentials, national energy and climate policies, and degree of interconnections, such sensitivity can allow for generalising the impact of 24/7 CFE procurement void regional differences.

\cref{fig:10-2025-DEPL-p3-4plots} shows the results for the two selected regions: Germany and Poland. 
Germany is selected because it is the largest electricity consumer in Europe and has a good connection to regions with relatively clean grids, like France and Denmark. 
As an example of the contrast, Poland is chosen because of its historical reliance on coal resources, resistance from the coal industry, and lack of government support for renewable energy.
The participation rate of 24/7~CFE consumers is set at 10\% of the \gls{ci} sector in each country, as in the base scenario.%
\footnote{10\% participation rate of \gls{ci} sector yields load of 3.8~GW in Germany and 1.1~GW in Poland.}

The dynamics of 24/7 CFE procurement, observed above, appear similar despite regional differences.

German consumers have lower average emissions rates due to a cleaner grid than Polish consumers, as one would expect.
In the reference case of grid electricity procurement, the values are at 240~g\co/kWh and 549~g\co/kWh for Germany and Poland, respectively.%
\footnote{For reference, 2021 data for emission intensity of electricity generation was at 402~g\co/kWh in Germany and 750~g\co/kWh Poland \cite{EEA-europa-web}.}
Similar to the Irish results, \cref{fig:10-2025-DE-p3-ci_emisrate} and \cref{fig:10-2025-PL-p3-ci_emisrate} show that in both cases, \gls{ci} participants achieve lower emissions rates with 24/7 hourly CFE procurement than with 100\% annual matching with renewable energy once their CFE targets are tight enough. 

A result that stands out in the cases of Germany and Poland is the impact of the 24/7 hourly CFE procurement on the system-level \co emissions, as shown in \cref{fig:10-2025-DE-p3-zone_emissions} and \cref{fig:10-2025-PL-p3-zone_emissions}.
Without any CFE procurement stategy, the German and Polish power sector carbon emissions are at 118.8~Mt\co and 83.8~Mt\co in 2025, respectively.%
\footnote{Germany is a larger electricity consumer, while Polish electricity mix has much a higher emission intensity; thus, power sector carbon emissions are on a similar scale for these countries.} 
If C\&I load commits to voluntary clean energy procurement beyond \gls{necp}s, a significant decarbonisation impact is achieved. 
The modelling results provides insightful data.
For example, in Germany, with just 10\% \gls{ci} participation rate, the 100\% annual matching with renewable energy can reduce national emissions by 11.1~Mt\co/a; the 24/7 hourly matching policy yields even greater impact, up to 14~Mt\co/a with CFE 100\% target.

The results for the required portfolio capacity-/costs are shown in \cref{fig:10-2025-DE-p3-ci_costandrev} and \cref{fig:10-2025-PL-p3-ci_costandrev}.
24/7 consumers in Germany and Poland complement procurement of renewable energy with electricity imports from grid for low CFE targets. 
To achieve higher CFE targets, Allam cycle with \gls{ccs} and advanced clean firm generators enter the technology mix. 
The fact that both advanced technologies enter the optimal solution space is different to Irish results, where only the Allam cycle is cost-optimal. 
This likely due to a case that Germany and Poland are better interconnected with the rest of the European electrcity grid than Ireland. 
A better interconnection makes possible to exhaust all arbitrage opportunities across time and space and maximise the dispatch value from an advanced dispatchable generator with low variable costs.
Similarly to Irish results though, the cost premium of 24/7 hourly matching is small when the advanced technologies are available.  

\begin{figure*}
    \centering
    \begin{subfigure}{0.5\textwidth}
        \centering
        \caption{Average emissions rate of participating consumers (Germany).}
        \includegraphics[width=0.95\columnwidth]{plots/10/2025/DE/p3/ci_emisrate.pdf}
        \label{fig:10-2025-DE-p3-ci_emisrate}
    \end{subfigure}%
    \begin{subfigure}{0.5\textwidth}
        \centering
        \caption{Average emissions rate of participating consumers (Poland).}
        \includegraphics[width=0.95\columnwidth]{plots/10/2025/PL/p3/ci_emisrate.pdf}
        \label{fig:10-2025-PL-p3-ci_emisrate}
    \end{subfigure}

    \begin{subfigure}{0.5\textwidth}
        \centering
        \caption{\co emissions in the local region (Germany).}
        \includegraphics[width=0.95\columnwidth]{plots/10/2025/DE/p3/zone_emissions.pdf}
        \label{fig:10-2025-DE-p3-zone_emissions}
    \end{subfigure}%
    \begin{subfigure}{0.5\textwidth}
        \centering
        \caption{\co emissions in the local region (Poland).}
        \includegraphics[width=0.95\columnwidth]{plots/10/2025/PL/p3/zone_emissions.pdf}
        \label{fig:10-2025-PL-p3-zone_emissions}
    \end{subfigure}%

    \begin{subfigure}{0.5\textwidth}
        \centering
        \caption{Costs of a procurement policy (Germany).}
        \includegraphics[width=0.95\columnwidth]{plots/10/2025/DE/p3/ci_costandrev.pdf}
        \label{fig:10-2025-DE-p3-ci_costandrev}
    \end{subfigure}%
    \begin{subfigure}{0.5\textwidth}
        \centering
        \caption{Costs of a procurement policy (Poland).}
        \includegraphics[width=0.95\columnwidth]{plots/10/2025/PL/p3/ci_costandrev.pdf}
        \label{fig:10-2025-PL-p3-ci_costandrev}
    \end{subfigure}

    \caption{Results for cases when 10\% of \gls{ci} load in Germany \textbf{(left)} and Poland \textbf{(right)} follows 24/7 hourly clean electricity procurement. Selected scenario with technological palette~3.}
    \label{fig:10-2025-DEPL-p3-4plots}
\end{figure*}


\subsection{Advancing five years forward}
\label{subsec:time}

\vspace{10pt}
\begin{res}
    In 2030, participating consumers benefit from decreasing costs of clean technologies and cleaner state of the electricity grids.
\end{res}

\vspace{5pt}
\begin{res}
    Voluntary commitments to clean electricity procurement bring system value even in
    cleaner grids. Similarly to the base scenario, 24/7 hourly matching results in deeper decarbonisation than 100\% annual renewable matching if CFE targets are above certain treshhold.
\end{res}

We explored the means, costs, and impacts of clean electricity procurement in the European electricity system for the year 2025 in the sections above.
In this section, we examine the results and implications of these procurement strategies for 2030.
Ultimately, this will allow us to better understand the long-term effects and the potential evolution of clean energy procurement in Europe, as companies continue to adapt their procurement strategies.

A five-year step to 2030 changes many system parameters from a modelling perspective.
In partcular, technology costs decline as a result of economies of scale and incremental innovation, energy- and climate policies become tighter, and some legacy power plants go out of business.
The two key trends of advancing five years forward are cleaner electricity grids and lower costs for energy technologies.

\cref{fig:10-2030-IE-6plots} illustrates the results for \textit{base scenario} moved five years to 2030, with all other factors held constant. 

\cref{fig:10-2030-IE-p1-used} shows the fraction of hourly demand met by clean electricity under each procurement policy in 2030.
Two observations can be drawn from comparing these results with a base scenario.
The first is that 72\% of demand is met by CFE in the reference case, which is 18\% greater than in 2025, i.e., there is a cleaner electricity grid on a background.
Therefore, lower CFE targets (such as 80\%) are achieved by importing more electricity.
Second, 24/7 participants continue to rely primarily on their own procurement of \gls{cfe} resources and storage, without importing much from the background grid to reach higher CFE targets.

A cleaner background grid results in lower average emissions rates for 24/7 participants. 
As shown in \cref{fig:10-2030-IE-p1-ci_emisrate}, the reference case value is reduced from 240~g\co/kWh in 2025 to 107~g\co/kWh in 2030.
The hourly matching policy allows achieving zero emissions associated with \gls{ci} participants consumption.
It is interesting to note that a CFE target of 90\% yields lower average emissions than a 100\% annual matching policy, even though these policies are both equally costly (see \cref{fig:10-2030-IE-p1-ci_costandrev}--\cref{fig:10-2030-IE-p3-ci_costandrev}).

In even cleaner systems, both 100\% annual matching with renewable energy and 24/7 hourly matching with \gls{cfe} keep their positive effect on system-level emissions.
Similar to 2025, when a CFE target is high enough, a diverse portfolio of energy resources procured by 24/7 participating consumers impacts system-level emissions more than the annual matching policy, see \cref{fig:10-2030-IE-p1-zone_emissions}.
The hourly matching policy nearly doubles the decarbonisation effect of renewable matching in 2030: with a 10\% participation rate, annual matching reduces local emissions by 0.2~Mt\co/a (approximately 8\% of Irish power section emissions). 
Hourly matching results in a reduction of 0.4~Mt\co/a at 100\% CFE target.
The nature of this effect is explained in \cref{subsec:mechanisms}.

The cost breakdown for Ireland 2030 shows that 24/7 hourly matching costs are decreasing for all technological pallets compared to 2025, see \cref{fig:10-2030-IE-p1-ci_costandrev}--\cref{fig:10-2030-IE-p3-ci_costandrev}.
The reason is that 24/7 participating consumers benefit from lower prices for clean energy technologies and cleaner electricity grids. 
The results show that cost-optimal portfolios procured by 24/7 participants for the 100\% hourly matching policy have a minor cost premium over the annual renewable matching, for scenarios when \gls{ldes} and advanced clean firm technologies are available. 
Through these low cost premiums, participating consumers achieve zero emissions associated with their consumption and have a great impact on system-level \co emissions.


\begin{figure*}
    \centering
    \begin{subfigure}{0.5\textwidth}
        \centering
        \caption{Fraction of hourly demand met with carbon-free electricity.}
        \includegraphics[width=0.95\columnwidth]{plots/10/2030/IE/p1/used.pdf}
        \label{fig:10-2030-IE-p1-used}
    \end{subfigure}%
    \begin{subfigure}{0.5\textwidth}
        \centering
        \caption{Average emissions rate of participating consumers.}
        \includegraphics[width=0.95\columnwidth]{plots/10/2030/IE/p1/ci_emisrate.pdf}
        \label{fig:10-2030-IE-p1-ci_emisrate}
    \end{subfigure}

    \begin{subfigure}{0.5\textwidth}
        \centering
        \caption{\co emissions in the local region (Ireland).}
        \includegraphics[width=0.95\columnwidth]{plots/10/2030/IE/p1/zone_emissions.pdf}
        \label{fig:10-2030-IE-p1-zone_emissions}
    \end{subfigure}%
    \begin{subfigure}{0.5\textwidth}
        \caption{Costs of a procurement policy: technological palette~1.}
        \includegraphics[width=0.95\columnwidth]{plots/10/2030/IE/p1/ci_costandrev.pdf}
        \label{fig:10-2030-IE-p1-ci_costandrev}
    \end{subfigure}%

    \begin{subfigure}{0.5\textwidth}
        \centering
        \caption{Costs of a procurement policy: technological palette~2.}
        \includegraphics[width=0.95\columnwidth]{plots/10/2030/IE/p2/ci_costandrev.pdf}
        \label{fig:10-2030-IE-p2-ci_costandrev}
    \end{subfigure}%
    \begin{subfigure}{0.5\textwidth}
        \centering
        \caption{Costs of a procurement policy: technological palette~3.}
        \includegraphics[width=0.95\columnwidth]{plots/10/2030/IE/p3/ci_costandrev.pdf}
        \label{fig:10-2030-IE-p3-ci_costandrev}
    \end{subfigure}

    \caption{Results for the case of Ireland 2030. 10\% participation rate. 
    Figures \ref{fig:10-2030-IE-p1-used}--\ref{fig:10-2030-IE-p1-ci_costandrev} display the technological palette~1 scenario.}
    \label{fig:10-2030-IE-6plots}
\end{figure*}



\subsection{Exploring mechanisms: how 24/7 CFE procurement decarbonises local grids}
\label{subsec:mechanisms}

\vspace{10pt}
\begin{res}
    24/7 CFE procurement reduces system-level \co \newline
    emissions through "profile" and "volume" mechanisms. 
\end{res}

\vspace{5pt}
\begin{res}
    The difference in impact that 24/7 CFE procurement and 100\% annual matching policies have on system decarbonisation amplifies with a cleaner state of local grids. %TODO REFINE
\end{res}

In the previous sections, we demonstrated that 24/7 clean energy procurement results in a substantial reduction of \co emissions in local electricity grids (Figures \ref{fig:10-2025-IE-p1-zone_emissions}, \ref{fig:10-2025-DE-p3-zone_emissions}, \ref{fig:10-2025-PL-p3-zone_emissions} and \ref{fig:10-2030-IE-p1-zone_emissions}).
In this section, we examine the underlying mechanisms that drive these observed reductions, so as to better understand how clean energy procurement and emissions reduction interrelate.

Two mechanisms are responsible for reducing system-wide emissions:

First, a \textit{profile} mechanism: participants doing 24/7 hourly matching procure clean energy resources that better match their demand patterns. 
As certain consumers timely align their demand with \gls{cfe} supply, the need for dispatchable generation that can firm intermittent renewable supply is lower in the rest of the system. 
This mechanism can reduce the utilisation of fossil-based generators, such as \gls{ocgt} power plants, which typically ramp up when wind and solar resources are scarce.

Second, a \textit{volume} mechanism: even the cost-optimal procurement strategy for 24/7 hourly matching might include some amount of excess clean energy.
If total \gls{cfe} generation of assets procured by 24/7 participants exceeds demand in a given hour and region, the “excess CFE” is not counted toward a CFE score; however, it is clean electricity that can potentially be stored (using batteries or LDES), or sold to the regional grid.
As for the latter, "excess CFE" might displace emitting grid generators or reduce the need to import electricity from neighbouring areas.

Several factors influence the degree to which profile and volume mechanisms contribute to decarbonising the local grids, such as the composition of the 24/7 portfolio, the volume of excess generation sold to the grid, and the electricity generation (and import) mix in the local zone.

In the following, we conduct an experiment to isolate the decarbonisation effects of the volume and profile mechanisms.
In the experiment, we reran the optimization with fixing an excess electricity sold to the grid to zero (NB excess is still possible, but can either be stored or curtailed). 
Such a model setup includes only a profile decarbonisation mechanism. 
We attribute the difference between emissions reduction in the experiment model setup and the base scenario to the volume mechanism.
As a result, the reduction of \co emissions in a local zone can be attributed either to excess electricity sold to the grid (volume effect) or to better alignment of CFE generation with demand (profile effect).

\cref{fig:10-profile-volume.pdf} illustrates the reduction in local zone emissions attributed to each of the two decarbonisation mechanisms.
In the plot, the x-axis shows the different states of electricity grids in which \gls{ci} consumers participate in 24/7 procurement.
The eight 'background grids' are generated from the combination of modelled wildcards: four zones and two years.

It can be seen in \cref{fig:10-profile-volume.pdf} that the overall impact of 24/7 CFE procurement is proportional to how clean the background grid on average is.
Thus, the largest impact is at 571.6~kgCO$_2$/a·MWh$^{-1}$ takes place in the Poland 2025 zone, and the lowest one is 3.7~kgCO$_2$/a·MWh$^{-1}$ in Denmark 2030 (NB 10\% of \gls{ci} sector participation rate with CFE 100\% score).
It is noteworthy that from 2025 to 2030, the decarbonisation effect of voluntary clean electricity procurement decreases over time in all zones since electricity grids become cleaner as a result of national climate policy programs and decomissioning of emitting power plants.
A comparison of the volume and profile mechanisms reveals that both contribute to the decarbonization of the local grids.
In particular, the profile mechanism contributes significantly to overall emissions reductions, with a major share across all zones regardless of their configuration.

Additional insights into decarbonisation mechanisms can be gained by examining the emission reductions achieved by 100\% renewables matching and 100\% 24/7 CFE matching policies.
\cref{fig:10-hourly-annual.pdf} compares the decarbonisation impact in a local zone of the two procurement policies in relative terms (percent point p.a.); the counterfactual scenario is a case when no \gls{ci} consumers participate in voluntary clean electricity procurement.
The figure shows that the relative impact of 24/7 CFE matching on system-level emissions \textit{increases over time}; this is despite the fact that the impact of procurement policies in absolute terms decreases over time, as shown above.
Furthermore, the self-cannibalisation effect of non-dispatchable generators, such as wind and solar, decreases the value of additional renewable capacity.
Thus, in the countries where governments commit to ambitious \gls{necp} targets, such as Ireland and Germany, 24/7 CFE matching has a stronger decarbonisation impact than 100\% annual matching with renewable energy. 
It is because the profile mechanism remains relevant even when renewable energy generators are self-cannibalizing; whereas the volume mechanism is amplified due to the diverse portfolio of procured resources, including batteries, long-term energy storage, and advanced clean firm technologies. As a result, clean electricity can be available during times when wind and solar generation is scarce system-wide.


\begin{figure}
    \centering
    \begin{subfigure}[t]{0.95\columnwidth}
        \centering
        \caption{Emissions reduction in a local zone with profile and volume mechanisms isolated. Data is normalised per MWh of demand participating in 24/7 procurement.}
        \includegraphics[width=0.95\columnwidth]{normalized_effects.pdf}
        \label{fig:10-profile-volume.pdf}
    \end{subfigure}
    \begin{subfigure}[t]{0.95\columnwidth}
        \centering
        \caption{Percentage of emissions reductions in a local zone as a result of a procurement policy compared with the no procurement policy case.}
        \includegraphics[width=0.95\columnwidth]{emissions_reduction_comparison.pdf}
        \label{fig:10-hourly-annual.pdf}
    \end{subfigure}
    \caption{
        System-level emission reduction: comparison of procurement policies and isolation of decarbonisation mechanisms.
        \cref{fig:10-profile-volume.pdf} shows breakdown of emissions reduction in a local zone if 10\% of \gls{ci} load follows 24/7 hourly procurement with CFE 100\% target. The profile are volume mechanisms are isolated.
        \cref{fig:10-hourly-annual.pdf} compares reductions in annual zone emissions in percent points achieved through 100\% annual renewable matching and 100\% 24/7 CFE matching; no procurement policy is assumed in the counterfactual scenario.}
    \label{fig:decarbonisation_story}
\end{figure}


% \subsection{System flexibility need reduction}
% \label{subsec:flexibility}

% \vspace{5pt}
% \begin{res}
%     24/7 CFE procurement reduces the needs for flexibility in the rest of the system.
% \end{res}
   
% The decarbonisation mechanisms discussed above suggest a displacement of fossil-fueled generators in the rest of the system in presense of voluntary clean electricity procurement.
% We illustrate this effect in \cref{fig:10-2030-DE-p1-system_capacity_diff.pdf}, which shows the difference in the generation capacity expansion of the whole European electricity system (excluding resources procured by participating customers) with and without commitments to clean electrity procurement (an illustraive scenario for Germany, 2030, technological palette~1). 
% Hourly CFE procurement benefits the system by having a pronounced effect of flexibility needs reduction -- the system requires less battery storage and less peaker capacity in a form of \gls{ocgt}. 
% In the selected example, 3.8~GW of load (10\% of \gls{ci} sectors in Germany) participate in the procurement.
% In a case of hourly matching with 100\% CFE target, this reduces \gls{ocgt} investments by 4.6~GW and battery storage by 2.6~GW in the rest of the system.
% The substitution of gas-fired capacity with clean resources procured by \gls{ci} consumers facilitates decarbonization of the entire system, as shown above.

% \begin{figure}
%     \centering
%     \includegraphics[width=0.95\columnwidth]{plots/10/2030/DE/p1/system_capacity_diff.pdf}
%     \caption{
%     Difference in generation capacity expansion in the whole European electricity system (including \gls{ci} resources) with and without clean energy procurement. 
%     Selected scenario of Germany, 2030, technological palette~1.}
%     \label{fig:10-2030-DE-p1-system_capacity_diff.pdf}
% \end{figure}
