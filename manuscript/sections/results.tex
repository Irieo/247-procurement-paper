
Throughout the analysis, we examine how voluntary clean electricity procurement strategies affect both buyers and the electricity system as a whole.
From the \gls{ci} buyers' perspective, we examine optimal procurement strategies (i.e., a mix of locally procured energy generation and storage assets and electricity imports), the average cost of procured electricity and the carbon emissions associated with electricity usage, also known as Scope 2 emissions \cite{GHGProtocolScope2}.
From a system perspective, we analyze how voluntary clean energy procurement impacts \co emissions in the region and the \enquote{demand pull} for advanced energy technologies.

Our experimental setup incorporates a wide range of scenarios, allowing us to generalise our results beyond individual assumptions.
For convenience, we start with a \textit{base scenario} presented in \cref{subsec:base}.
The base scenario reflects a case when 10\% (on volumetric basis) of \gls{ci} consumers in \textit{Ireland} in \textit{2025} are committed to procuring additional clean energy with a desired strategy.
The participating consumers can attain the 24/7~CFE procurement targets by co-optimizing local procurement using mature technologies, designated as \textit{palette~1}, and by importing electricity from the local grid. Afterward, we delve deeper into the scenario space, altering selected parameters to explore various alternatives and address the following questions:

\begin{enumerate}[-]
\item \cref{subsec:palette}: What if a wider palette of carbon-free technologies were available to participating buyers?
\item \cref{subsec:location}: What if participating buyers are located in areas of the European electricity grid that differ significantly from the base scenario?
\item \cref{subsec:time}: What are the implications if we move the timeframe ahead five years to 2030?
\item Finally, in \cref{subsec:mechanisms}, we get a closer look at how 24/7 CFE commitments can help decarbonize local grids and reveal the individual mechanisms involved.
\end{enumerate}

Considering the large scenario space, this work presents results for a selection of scenario branches that generalise the system impacts of 24/7 CFE procurements.
We publish an online dashboard with results for the entire scenario space alongside the paper, see \nameref{sec:code}.

\subsection{Base scenario}
\label{subsec:base}

\begin{res}
    Through 24/7 CFE procurement, participating consumers can eliminate carbon emissions associated with their electricity consumption and help to decarbonise the entire electricity system.
\end{res}

\begin{res}
    90\%--95\% 24/7 CFE targets have only a small cost premium over 100\% annual renewable matching. 100\%~CFE target can be achieved with solar PV, wind and battery storage, but at a high cost premium for participating consumers.
\end{res}

%some parts of this paragraph may go to methods
\cref{fig:10-2025-IE-p1-used} shows the fraction of hourly demand met with \gls{cfe} depending on the procurement policy that participating consumers follow.
We contrast two main cases, one where participants commit to 100\% annual renewable matching policy (hereinafter -- \enquote{100\% RES}) and another case where the consumers commit to 24/7~CFE hourly matching policy (hereinafter -- \enquote{24/7~CFE}) with various CFE targets in a range from 80\% to 100\%. 
In the \textit{reference case}, participating consumers cover their load purely with grid purchases without any policy regarding the origin of electricity.

\begin{figure}
    \centering
    \includegraphics[width=0.95\columnwidth]{plots/10/2025/IE/p1/used.pdf}
    \caption{Fraction of hourly demand met with carbon-free electricity.}
    \label{fig:10-2025-IE-p1-used}
\end{figure}

In the reference case, only 54\% of demand is met with \gls{cfe}.
100\% RES---the best case for the annual renewable matching policy---results in a 83\% fraction.
Consequently, CFE targets beyond 85\% yield higher share of hourly demand met with \gls{cfe} than 100\%~RES.
Another notable trend in \cref{fig:10-2025-IE-p1-used} is that participating consumers rely more on procured resources and less on grid imports with higher CFE targets.
When targets approach 100\%, participating consumers cannot rely on electricity imports from the grid, since the grid supply mix contains some carbon content in almost all hours.

\begin{figure}[t]
    \centering
    \begin{subfigure}[t]{0.95\columnwidth}
        \centering
        \caption{Average emissions rate of participating consumers.}
        \includegraphics[width=0.95\columnwidth]{plots/10/2025/IE/p1/ci_emisrate.pdf}
        \label{fig:10-2025-IE-p1-ci_emisrate}
    \end{subfigure}
    \begin{subfigure}[t]{0.95\columnwidth}
        \centering
        \vspace{-0.2cm}
        \caption{\co emissions in the local region (Ireland).}
        \includegraphics[width=0.95\columnwidth]{plots/10/2025/IE/p1/zone_emissions.pdf}
        \label{fig:10-2025-IE-p1-zone_emissions}
    \end{subfigure}
    \caption{Impact of clean electricity procurement commitments on emissions of participating consumers (top panel) and the system-level emissions in a local region (bottom panel).}
    \label{fig:10-2025-IE-p1-emissions}
\end{figure}

\cref{fig:10-2025-IE-p1-ci_emisrate} shows how choice of a procurement policy affects the average emissions rate of participating consumers.
Already in 2025, Ireland has a moderately clean electricity system: the emission rate is at 177~g\co/kWh in the reference case.
If participating consumers follow 100\%~RES policy, the emission rate is reduced to 60~g\co/kWh.
This yields a considerable emission rate reduction compared to the reference case.
The 24/7~CFE procurement with targets above 85\% helps the participating consumers to reduce emissions further beyond the treshhold set by the 100\%~RES. 
At the 100\% CFE target, participating consumers attain a zero emission rate associated with their electricity consumption, as they perfectly align demand and \gls{cfe} supply on an hourly basis.

Voluntary procurement of clean electricity also helps decarbonise the electricity system as a whole.
We explore this effect by plotting \co emissions in the local region of the participating consumers, i.e. Ireland (\cref{fig:10-2025-IE-p1-zone_emissions}).
Without voluntary procurement, the model estimates Irish power sector carbon emissions to be at the level of 4.2 Mt\co/a.\footnote{For comparison, \href{https://www.seai.ie/data-and-insights/seai-statistics/key-publications/co2-emissions-report/}{seai.ie} reports this value to be at 10.3~Mt\co in 2021. A strong decreasing trend is expected, since the Irish government has set ambitious targets to achieve a low-carbon energy system \cite{seaiCOEmissionsReport2020}.}
The modelling results indicate that a commitment to 100\% 24/7~CFE in 2025 from 10\% of the \gls{ci} electricity demand could yield a reduction in Irish emissions by 0.6~Mt\co/a, in contrast to a scenario absent voluntary procurement. 
This reduction equates to 14\% of the total emissions from the Irish power sector.
100\%~RES can also facilitate system-level \co emissions reductions (as depicted by the left bar in \cref{fig:10-2025-IE-p1-zone_emissions}); however, 24/7~CFE attains superior emissions reductions when the CFE target surpasses 85\%.


\begin{figure}[t]
    \centering
    \begin{subfigure}[t]{0.95\columnwidth}
        \centering
        \caption{Portfolio capacity procured by participating consumers.}
        \includegraphics[width=0.95\columnwidth]{plots/10/2025/IE/p1/ci_capacity.pdf}
        \label{fig:10-2025-IE-p1-ci_capacity}
    \end{subfigure}
    \begin{subfigure}[t]{0.95\columnwidth}
        \centering
        \vspace{-0.2cm}
        \caption{Cost breakdown of a procurement policy.}
        \includegraphics[width=0.95\columnwidth]{plots/10/2025/IE/p1/ci_costandrev.pdf}
        \label{fig:10-2025-IE-p1-ci_costandrev}
    \end{subfigure}
    \caption{Optimal capacity portfolios (top panel) and costs breakdown (bottom panel) per procurement strategy.}
    \label{fig:10-2025-IE-p1-ci_procurement}
\end{figure}

The following analysis unveils the modelled cost-optimal procurement strategies for each procurement policy (\cref{fig:10-2025-IE-p1-ci_capacity}).
In Ireland, 10\% of the \gls{ci} sector's demand yields a load of 220~MW.
The findings indicate that, to cover this load adhering to a 100\%~RES policy, participating consumers procure a combined capacity of 1.3~GW from onshore wind and solar PV generators.
Ensuring demand is matched with \gls{cfe} on an hourly basis requires a considerably more substantial portfolio of wind and solar PV, up to 4.0~GW of combined capacity to reach the 100\%~CFE target.
Batteries are integrated into the cost-optimal portfolio mix when CFE targets exceed 85\%.
It is noteworthy that, for an 80\%~CFE target, participating consumers procure less capacity than under a 100\%~RES policy, as they rely more on grid imports.

The breakdown of costs associated with a procurement policy that participating consumers choose is shown in \cref{fig:10-2025-IE-p1-ci_costandrev}.
Note that revenues from selling the excess electricity to the regional grid at market prices can be treated as \enquote{negative costs} and subtracted from the net procurement cost.
The results show that a CFE targets of 80\%--95\% can be achieved at a small cost premium to 100\%~RES policy with solar, wind and batteries.
However, what stands out in the plot is the rapid increase of procurement costs for high 24/7 CFE targets. 
For example, 98\% CFE target has cost premium of 54\% over 100\% the annual matching policy; while the last 2\% of hourly matching more than doubles the costs.


\subsection{Technology access}
\label{subsec:palette}

\begin{res}
    24/7 CFE procurement creates an early market for advanced energy technologies the system will need later: long-duration energy storage and clean firm generators.
\end{res}

\begin{res}
    The cost premium of 24/7 procurement with high CFE targets may substantially decrease if advanced energy technologies are available for commercial scale-up. 
\end{res}

The cost premium of 24/7~CFE procurement presented in \cref{subsec:base} is driven by the variability of renewable power supply.
Indeed, in the periods when not much wind or solar is available, matching every kWh of electricity consumption with carbon-free electricity on hourly basis is not an easy task.
Battery storage technology is cost-optimal for shifting surplus power supply by a couple of hours.
However, bridging extended periods of low wind and sun with battery storage is expensive.
Also, the results above show that 24/7 participating consumers rely little on grid supply to reach CFE targets of 98\%--100\%, since a local grid's electricity mix almost always includes some fossil-fueled generation.

The results in \cref{fig:10-2025-IE-p23-ci_procurement} reflect a case when 24/7~CFE participants have access to a wider palette of technologies, including \gls{ldes} and advanced clean firm generators, such as advanced geothermal systems, advanced nuclear or Allam cycle generators with \gls{ccs}.

\cref{fig:10-2025-IE-p2-ci_costandrev} (left panel) shows the breakdown of costs for a case when participating consumers have an access to palette~2 technologies, i.e., long-duration storage complements onshore wind, solar PV and battery storage.
Here we assume an underground hydrogen storage in salt caverns with energy storage cost of 2.5~\euro/kWh [\cref{tab:tech_costs}].
An \gls{ldes} system helps to bridge hours with no renewable feed-in. 
As a result, the required portfolio of renewable capacity and procurement costs are reduced, especially for high CFE targets.
Thus, 100\% 24/7~CFE policy has a cost premium of 42\% above 100\%~RES, which is significantly lower than without \gls{ldes}.

\cref{fig:10-2025-IE-p3-ci_costandrev} (right panel) depicts the technological palette~3 scenario, wherein participating consumers additionally gain access to clean firm generation technologies.
The Allam cycle turbine with \gls{ccs} now enters into the optimal procurement strategy.
The cost-optimal share of to clean firm technology increases as CFE targets rise.
The clean dispatchable generator helps to fill in the gaps in variable renewable generation to match demand and \gls{cfe} supply on an hourly basis.
The cost premium is further reduced: 100\% 24/7 CFE costs only 15\% above the 100\%~RES.

\begin{figure*}
    \centering
    \begin{subfigure}{0.5\textwidth}
        \centering
        \caption{Cost breakdown of a procurement policy (w/ \gls{ldes}).}
        \includegraphics[width=0.95\columnwidth]{plots/10/2025/IE/p2/ci_costandrev.pdf}
        \label{fig:10-2025-IE-p2-ci_costandrev}
    \end{subfigure}%
    \begin{subfigure}{0.5\textwidth}
        \centering
        \caption{Costs of a procurement policy \\ 
        (w/ \gls{ldes} and advanced technologies).}
        \includegraphics[width=0.95\columnwidth]{plots/10/2025/IE/p3/ci_costandrev.pdf}
        \label{fig:10-2025-IE-p3-ci_costandrev}
    \end{subfigure}
    \caption{The breakdown of costs per procurement policy if participating consumers have an access to a wider palette of technologies: w/ \gls{ldes} (left panel); w/ \gls{ldes} and advanced clean firm generators (right panel).
    }
    \label{fig:10-2025-IE-p23-ci_procurement}
\end{figure*}


\subsection{Diverse grid locations}
\label{subsec:location}

\begin{res}
    Despite the inherent variability in regional charac\-teristics---ranging from renewable resources and national policies to degrees of interconnection---the model shows a consistent effects of 24/7 clean energy procurement on both participating buyers and overarching system-level impacts across all analyzed regions.
\end{res}

While the primary focus has been a base scenario exemplifying Ireland, analogous modelling can be applied to diverse regions within the European electricity grid, each exhibiting unique characteristics influenced by local resources, renewable potentials, \gls{necp}s, the degree of interconnections, and other factors. 

\cref{fig:10-2025-DEPL-p3-4plots} depicts outcomes for the selected regions of Germany and Poland. 
Germany is selected due to its extensive interconnections with neighbouring regions with relatively clean grids, such as France and Denmark, and its status as the largest electricity consumer in Europe. 
In contrast, Poland has a smaller electricity demand. 
Moreover, the Polish energy system typically has higher carbon intensity, largely due to its historical dependence on coal resources.
The participation rate is maintained at 10\% of the \gls{ci} sector in both regions, aligning with the base scenario.\footnote{A 10\% participation rate within the \gls{ci} sector corresponds to loads of 3.8~GW in Germany and 1.1~GW in Poland.}
Despite regional variances, the dynamics and system impacts of clean energy procurement, as observed in the base scenario, exhibit similarities.

German consumers experience lower average emissions rates, attributed to a cleaner grid, in contrast to those in Poland.
In the reference cases, emission values stand at 240~g\co/kWh and 549~g\co/kWh for Germany and Poland, respectively.\footnote{For context, the emission intensity of electricity generation in 2021 was 402~g\co/kWh for Germany and 750~g\co/kWh for Poland \cite{EEA-GNGEmissionsEU}.} 
Echoing the Irish scenario, both \cref{fig:10-2025-DE-p3-ci_emisrate} and \cref{fig:10-2025-PL-p3-ci_emisrate} demonstrate that in both contexts, participating consumers attain reduced emissions rates with 24/7~CFE procurement compared to 100\%~RES, given sufficiently strict CFE targets.

The impact of 24/7 hourly CFE procurement on system-level \co emissions is notable in both Germany and Poland, as depicted in \cref{fig:10-2025-DE-p3-zone_emissions} and \cref{fig:10-2025-PL-p3-zone_emissions}.
In the absence of any procurement strategy, carbon emissions from the power sectors of Germany and Poland are projected to be 118.8~Mt\co and 83.8~Mt\co in 2025, respectively.%
\footnote{Despite Germany consuming more electricity, the similar scale of power sector carbon emissions between the two countries can be attributed to Poland's significantly higher emission intensity in its electricity mix.} 
By engaging in voluntary clean energy procurement, \gls{ci} consumers can drive notable decarbonisation in the local regions, exceeding the benchmarks set by \gls{necp}s.
For instance, in Germany, even a 10\% participation rate under a 100\%~RES policy can diminish national emissions by 11.1~Mt\co annually; remarkably, the 24/7~CFE approach can amplify this impact to up to 14~Mt\co/a, given a CFE target of 100\%. 

The optimal capacities of energy technologies and associated costs are illustrated in \cref{fig:10-2025-DE-p3-ci_costandrev} and \cref{fig:10-2025-PL-p3-ci_costandrev}.
Similar to the Irish scenario, participating consumers complement renewable energy procurement with electricity imports from the grid to meet lower CFE targets. 
For the realization of strict CFE targets, both the Allam cycle with \gls{ccs} and advanced clean firm generators are incorporated into the technology mix. 
This entry of two advanced technologies into the optimal solution space diverges from the Irish findings, where solely the \gls{opex}-heavy Allam cycle turbine is deemed cost-optimal. 
The difference is likely attributable to Germany and Poland being more extensively interconnected with the broader European electricity grid compared to Ireland. 
Better interconnection allows exhausting all arbitrage opportunities across temporal and spatial dimensions, and maximising the dispatch value derived from an advanced dispatchable generator with high \gls{capex}.
Mirroring the findings from Ireland, the cost premium of 24/7~CFE matching gets minimal when these advanced technologies are accessible.

\begin{figure*}
    \centering
    \begin{subfigure}{0.5\textwidth}
        \centering
        \caption{Average emissions rate of participating consumers (Germany).}
        \includegraphics[width=0.95\columnwidth]{plots/10/2025/DE/p3/ci_emisrate.pdf}
        \label{fig:10-2025-DE-p3-ci_emisrate}
    \end{subfigure}%
    \begin{subfigure}{0.5\textwidth}
        \centering
        \caption{Average emissions rate of participating consumers (Poland).}
        \includegraphics[width=0.95\columnwidth]{plots/10/2025/PL/p3/ci_emisrate.pdf}
        \label{fig:10-2025-PL-p3-ci_emisrate}
    \end{subfigure}

    \begin{subfigure}{0.5\textwidth}
        \centering
        \caption{\co emissions in the local region (Germany).}
        \includegraphics[width=0.95\columnwidth]{plots/10/2025/DE/p3/zone_emissions.pdf}
        \label{fig:10-2025-DE-p3-zone_emissions}
    \end{subfigure}%
    \begin{subfigure}{0.5\textwidth}
        \centering
        \caption{\co emissions in the local region (Poland).}
        \includegraphics[width=0.95\columnwidth]{plots/10/2025/PL/p3/zone_emissions.pdf}
        \label{fig:10-2025-PL-p3-zone_emissions}
    \end{subfigure}%

    \begin{subfigure}{0.5\textwidth}
        \centering
        \caption{Cost breakdown of a procurement policy (Germany).}
        \includegraphics[width=0.95\columnwidth]{plots/10/2025/DE/p3/ci_costandrev.pdf}
        \label{fig:10-2025-DE-p3-ci_costandrev}
    \end{subfigure}%
    \begin{subfigure}{0.5\textwidth}
        \centering
        \caption{Cost breakdown of a procurement policy (Poland).}
        \includegraphics[width=0.95\columnwidth]{plots/10/2025/PL/p3/ci_costandrev.pdf}
        \label{fig:10-2025-PL-p3-ci_costandrev}
    \end{subfigure}
    \caption{Selected results for scenarios when participating consumers are located in Germany (left) and Poland (right); all plots are for technological palette~3.} 
    \label{fig:10-2025-DEPL-p3-4plots}
\end{figure*}


\subsection{Advancing five years forward}
\label{subsec:time}

\begin{res}
    Over time, participating consumers reap benefits from the gradual decline in clean technology costs and a progressively cleaner state of the electricity grids, enhancing the affordability of 24/7 CFE procurement.
\end{res}

\begin{res}
    Voluntary commitments to 24/7 CFE procurement retain their significance in enhancing system value, even as grids become cleaner. Analogous to the base scenario, hourly matching commitments facilitate a more profound decarbonisation compared to 100\% annual renewable matching, provided that CFE targets surpass a particular threshold.
\end{res}

Having explored the methods, costs, and impacts of clean electricity procurement in the European electricity system for the year 2025 in preceding sections, this section delves into the projected results and implications of these strategies for 2030.
Referencing back to the \nameref{sec:methods} section, the key implications of advancing five years forward are cleaner electricity grids and decreased costs for energy  technologies. \cref{fig:10-2030-IE-6plots} illustrates the outcomes for the \textit{base scenario} transitioned to 2030, maintaining all other parameters constant. 

Analyzing \cref{fig:10-2030-IE-p1-used}, which presents the fraction of hourly demand met by clean electricity under each procurement policy in 2030, two key insights emerge when compared to the base scenario.
Firstly, 72\% of demand is met by \gls{cfe} in the reference case, marking an 18\% increase from 2025, indicating an enhancement in the cleanliness of the background electricity grid.
Consequently, lower CFE targets (like 80\%) are met with higher share of electricity imports.
Secondly, participating consumers depend on their own procurement of \gls{cfe} resources and storage, limiting imports from the background grid to achieve strict CFE targets. 
This effect stays consistent with the base scenario.

A cleaner background grid results in lower average emissions rates for participating consumers.
\cref{fig:10-2030-IE-p1-ci_emisrate} shows that the emissions rate for consumers purchasing only grid electricity decreases from 240~g\co/kWh in 2025 to 107~g\co/kWh in 2030.
The hourly matching policy enables the attainment of zero emissions linked to \gls{ci} participants' consumption.
It is interesting to note that a CFE target of 90\% yields lower average emissions than a 100\% annual matching policy, even though these policies are equally costly (see \cref{fig:10-2030-IE-p1-ci_costandrev}--\cref{fig:10-2030-IE-p3-ci_costandrev}).

In cleaner systems, both 100\%~RES and 24/7~CFE policies sustain their beneficial impact on system-level emissions. 
As observed in \cref{fig:10-2030-IE-p1-zone_emissions}, 24/7~CFE with a sufficiently high CFE target, outperforms 100\%~RES policy in the decarbonisation impact: with 10\% participation rate, hourly matching reduces local emissions by up to 0.4~Mt\(\text{CO}_2\)/a, in contrast to the 0.2~Mt\(\text{CO}_2\)/a reduction achieved by annual matching. 
The 0.2~Mt\(\text{CO}_2\)/a differential equates to approximately 8\% of Irish power sector emissions.
The mechanisms driving this impact are detailed in \cref{subsec:mechanisms}.

As illustrated in \cref{fig:10-2030-IE-p1-ci_costandrev}--\cref{fig:10-2030-IE-p3-ci_costandrev}, the cost breakdown of the procurement policy reveals a decline in 24/7~CFE costs across all technological pallets relative to 2025.
Participating consumers benefit from lower prices for clean energy technologies and progressively cleaner electricity grids. 
Remarkably, these participants not only achieve zero emissions associated with their consumption and exert a notable impact on system-level \(\text{CO}_2\) emissions but also do so while encountering smaller cost premiums, enhancing the accessibility of the voluntary clean energy procurement goals.


\begin{figure*}
    \centering
    \begin{subfigure}{0.5\textwidth}
        \centering
        \caption{Fraction of hourly demand met with carbon-free electricity.}
        \includegraphics[width=0.95\columnwidth]{plots/10/2030/IE/p1/used.pdf}
        \label{fig:10-2030-IE-p1-used}
    \end{subfigure}%
    \begin{subfigure}{0.5\textwidth}
        \centering
        \caption{Average emissions rate of participating consumers.}
        \includegraphics[width=0.95\columnwidth]{plots/10/2030/IE/p1/ci_emisrate.pdf}
        \label{fig:10-2030-IE-p1-ci_emisrate}
    \end{subfigure}

    \begin{subfigure}{0.5\textwidth}
        \centering
        \caption{\co emissions in the local region (Ireland).}
        \includegraphics[width=0.95\columnwidth]{plots/10/2030/IE/p1/zone_emissions.pdf}
        \label{fig:10-2030-IE-p1-zone_emissions}
    \end{subfigure}%
    \begin{subfigure}{0.5\textwidth}
        \caption{Cost breakdown of a procurement policy: technological palette~1.}
        \includegraphics[width=0.95\columnwidth]{plots/10/2030/IE/p1/ci_costandrev.pdf}
        \label{fig:10-2030-IE-p1-ci_costandrev}
    \end{subfigure}%

    \begin{subfigure}{0.5\textwidth}
        \centering
        \caption{Cost breakdown of a procurement policy: technological palette~2.}
        \includegraphics[width=0.95\columnwidth]{plots/10/2030/IE/p2/ci_costandrev.pdf}
        \label{fig:10-2030-IE-p2-ci_costandrev}
    \end{subfigure}%
    \begin{subfigure}{0.5\textwidth}
        \centering
        \caption{Cost breakdown of a procurement policy: technological palette~3.}
        \includegraphics[width=0.95\columnwidth]{plots/10/2030/IE/p3/ci_costandrev.pdf}
        \label{fig:10-2030-IE-p3-ci_costandrev}
    \end{subfigure}

    \caption{Results for the scenario of Ireland 2030; 10\% participation rate. 
    Figures \ref{fig:10-2030-IE-p1-used}--\ref{fig:10-2030-IE-p1-ci_costandrev} display the technological palette~1 scenario.}
    \label{fig:10-2030-IE-6plots}
\end{figure*}



\subsection{Understanding the mechanism of 24/7 CFE procurement in grid decarbonisation}
\label{subsec:mechanisms}

\begin{res}
    24/7 CFE procurement mitigates system-level \co \newline 
    emissions through two distinct mechanisms: \enquote{profile} and \enquote{volume}. The mechanisms origin from distinct aspects of the interplay between the 24/7 CFE procurement and the background electricity grids.
\end{res}

\begin{res}
    The disparity in decarbonisation outcomes between 24/7 CFE hourly and 100\% annual renewable matching policies becomes increasingly pronounced as local grids transition to cleaner states. This phenomenon underscores the effectiveness of 24/7 CFE procurement in supporting system decarbonisation in the context of evolving national energy and climate policies.
\end{res}

In the previous sections, we demonstrated that 24/7 clean energy procurement results in a substantial reduction of \co emissions in local electricity grids (Figures \ref{fig:10-2025-IE-p1-zone_emissions}, \ref{fig:10-2025-DE-p3-zone_emissions}, \ref{fig:10-2025-PL-p3-zone_emissions} and \ref{fig:10-2030-IE-p1-zone_emissions}).
In this section, we examine the underlying mechanisms that drive these observed reductions, so as to better understand how clean energy procurement and emissions reduction interrelate.

As originally identified by Xu et al. (2021) \cite{xu-247CFE-report}, two mechanisms are responsible for reducing system-wide emissions:

First, a \textit{profile} mechanism: participants doing 24/7 hourly matching procure clean energy resources that better match their demand patterns. 
As certain consumers align their demand with \gls{cfe} supply on hourly basis, the need for dispatchable generation that can firm intermittent renewable supply is lower in the rest of the system. 
This mechanism can reduce the utilisation of fossil-based generators, such as \gls{ocgt} power plants, which typically ramp up when wind and solar resources are scarce.

Second, a \textit{volume} mechanism: even the cost-optimal procurement strategy for 24/7 hourly matching might include some amount of excess clean energy.
If total \gls{cfe} generation of assets procured by 24/7 participants exceeds demand in a given hour, the \enquote{excess CFE} is not counted toward a CFE target; however, it is clean electricity that can potentially be stored (using batteries or LDES), or sold to the regional grid.
As for the latter, excess CFE might displace emitting grid generators or reduce the need to import electricity from neighbouring areas.

Several factors influence the degree to which profile and volume mechanisms contribute to decarbonising the local grids, such as the composition of the 24/7 portfolio, the volume of excess generation sold to the grid, and the electricity generation (and import) mix in the local zone.

We conduct an experiment to isolate the decarbonisation effects of the volume and profile mechanisms.
In the experiment, we reran the optimization while fixing an excess electricity sold to the grid to zero (NB excess is still possible, but must be either stored or curtailed). 
Such a model setup includes only a profile decarbonisation mechanism. 
We attribute the difference between emissions reduction in the experiment model setup and the base scenario to the volume mechanism.
As a result, the reduction of emissions in a local zone can be attributed either to excess electricity sold to the grid (volume effect) or to better alignment of CFE generation with demand (profile effect).

\cref{fig:10-profile-volume.pdf} illustrates the reduction in local zone emissions attributed to each of the two decarbonisation mechanisms.
In the plot, the x-axis shows the different states of electricity grids in which \gls{ci} consumers participate in 24/7 procurement.
The eight 'background grids' are generated from the combination of modelled wildcards: four zones and two years.

\cref{fig:10-profile-volume.pdf} shows that the impact of 24/7 CFE procurement in absolute terms is proportional to how clean the background grid on average is.
Thus, the largest impact is at 571.6~kgCO$_2$/a·MWh$^{-1}$ takes place in the Poland 2025 zone, and the lowest one is 3.7~kgCO$_2$/a·MWh$^{-1}$ in Denmark 2030 (NB values for 10\% of \gls{ci} sector participation rate with CFE 100\% target).
It is noteworthy that from 2025 to 2030, the decarbonisation effect of voluntary clean electricity procurement decreases over time in all zones since electricity grids become cleaner as a result of national climate policy programs and decomissioning of emitting power plants.
A comparison of the volume and profile mechanisms reveals that both contribute to the decarbonization of the local grids.
In particular, the profile mechanism contributes significantly to overall emissions reductions, with a major share across all zones regardless of their configuration.

\begin{figure}[H]
    \centering

    \begin{subfigure}[t]{\columnwidth}
        \centering
        \caption{Emissions reduction in a local zone with profile and volume mechanisms isolated. Data is normalised per MWh of demand participating in 24/7 procurement.}
        \includegraphics[width=0.95\columnwidth]{normalized_effects.pdf}
        \label{fig:10-profile-volume.pdf}
    \end{subfigure}

    \begin{subfigure}[t]{\columnwidth}
        \centering
        \caption{Percentage of emissions reductions in a local zone as a result of a procurement policy compared with the no procurement policy case.}
        \includegraphics[width=0.95\columnwidth]{emissions_reduction_comparison.pdf}
        \label{fig:10-hourly-annual.pdf}
    \end{subfigure}

    \caption{
        System-level emission reduction: comparison of procurement policies and isolation of decarbonisation mechanisms.
        \cref{fig:10-profile-volume.pdf} shows breakdown of emissions reduction in a local zone if 10\% of \gls{ci} load follows 24/7 hourly procurement with CFE 100\% target. The profile are volume mechanisms are isolated.
        \cref{fig:10-hourly-annual.pdf} compares reductions in annual zone emissions in percent points achieved through 100\% annual renewable matching and 100\% 24/7 CFE matching; no procurement policy is assumed in the counterfactual scenario.}
        \label{fig:decarbonisation_story}

\end{figure}


Additional insights into decarbonisation mechanisms can be gained by examining the emission reductions achieved through 100\%~RES and 100\% 24/7 CFE matching policies.
\cref{fig:10-hourly-annual.pdf} compares the decarbonisation impact within a local zone of the two procurement policies in relative terms (percent point p.a.), with a counterfactual scenario where no consumers engage in voluntary clean electricity procurement.
The figure demonstrates that, even though the absolute impact of procurement policies diminishes over time, the relative impact of 24/7 CFE on system-level emissions \textit{increases over time} when compared to 100\%~RES.
This phenomenon is attributable to the self-cannibalisation effect of wind and solar that influences the value of additional renewable capacity.
This is relevant in regions where governments adhere to ambitious \gls{necp} targets, like Ireland and Germany (NB Denmark is an outlier with an exceptional RES target exceeding 100\% of national electricity demand for 2030).
The profile mechanism maintains its relevancy even when renewable energy generators undergo self-cannibalization.
The volume mechanism retains, or even amplifies, its relevance, by leveraging a diversified portfolio of procured resources by 24/7 participants, including batteries, long-term energy storage, and advanced clean firm technologies.
Consequently, clean electricity is not only available but can also be sold to the background grid during periods when system-wide wind and solar generation is scarce.
