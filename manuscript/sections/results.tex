We explore how the 247-CFE and the 100\% annual renewable matching procurement approaches affect both the participating consumers (driving the average costs of procured electricity and carbon emissions associated with the consumers' electricity usage) and the electricity system as a whole. 
Regarding the latter one, we examine both the short-term impacts (market dispatch, revenue streams for electricity generators, system emissions, curtailment) and the long-term impacts (exit of fossil-fueled generators from the market, improved market conditions for storage and new clean dispatchable technologies like geothermal power or advanced nuclear technologies).

\subsection{title}

text 

\vspace*{5pt}
\begin{res}
    24/7-CFE procurement reduces emissions both for the participating electricity consumer and within the rest of the electricity system.
\end{res}

\vspace*{5pt}
\begin{res}
    24/7-CFE procurement comes with a cost premium. However, the 80-90\% hourly CFE target has only a small cost premium over 100\% annual renewable matching. 100\% hourly CFE target increases costs by 0-60\%, depending on technologies available.
\end{res}

\vspace*{5pt}
\begin{res}
    24/7-CFE procurement triggers investment in new technologies the system will need later: long-duration storage and new clean firm generation.
\end{res}


\subsection{title}

text
