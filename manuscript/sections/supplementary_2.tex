%S2: Background system assumptions

The model performs \textit{a perfect-foresight optimisation} of investment and power dispatch decisions to meet electricity demand of the 24/7 consumers, as well as the demand of other consumers in the European electricity system for 2025 or 2030.

In all the modelled countries, renewable generation \textit{must meet the political targets} as defined in the \gls{necp}s or by more recent national policy targets (such as the Easter package in Germany). In countries that not have a target for 2025, a linear increase from targets between 2020 and 2030 is assumed. The assumed renewable targets are available in the config file of the scientific workflow \cite{github-247CFEpaper}.

We assume 2013 to be a representative climate year for renewable in-feed. The renewable feed-in profiles are computed with the atlite package \cite{atlite-github}.

The country demand profiles for 2025 and 2030 are assumed to be the same as in 2013. Time-series data for electricity demand is based on the Open Power System Data project \cite{OPSD}.

The existing power plant fleet data is based on the powerplantmatching package \cite{Powerplantmatching-github}. We consider national policies and decommissioning plans for coal and nuclear power plants based on data from the \enquote{beyondfossilfuels.org} and the \enquote{world-nuclear.org} projects \cite{FossilFuels, WorldNuclearAssociation}.

We assume price for EU ETS allowances to be 80~\euro/t\co and 130~\euro/t\co  for 2025 and 2030, accordingly. 
The price for natural gas is assumed to be 35~\euro/MWh.