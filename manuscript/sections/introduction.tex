% Introduction Structure

% 1. **Opening & Introduce the global momentum towards renewable energy adoption.**

Climate change is driving a global effort to rapidly decarbonise electricity systems across the globe. 
The European Union (\gls{eu}) has committed to achieving net-zero emissions by 2050, and the European Commission has adopted the European Climate Law with a target of reducing net greenhouse gas emissions by at least 55\% by 2030, compared to 1990 levels \cite{EuropeanClimateLaw2020}. 
National climate initiatives are gaining ground across the globe, including the United States with the Biden administration's goal of reducing emissions by 50-52\% by 2030, compared to 2005 levels \cite{BidenClimatePlan2021}, and China's commitment to have \co emissions peak before 2030 and achieve carbon neutrality before 2060 \cite{ChinaNetZero-IEA}.

% 2. **Evolution, approaches and challenges of voluntary clean energy procurement**

Many public and private energy buyers have joined this effort by purchasing clean energy to demonstrate their sustainability credentials. 
The established certification schemes for voluntary clean energy procurement are the Guarantees of Origin (\gls{go}s) in Europe and the Renewable Energy Certificates (\gls{rec}s) in the United States.
A common feature of these schemes is that renewable energy credits are \enquote{unbundled} from megawatt-hours of energy and can be traded numerous times on specialized markets before being retired.
The owner gets to claim environmental benefits when \enquote{retiring} (i.e. using) the certificate.
The unbundled renewable energy credits, however, have been criticized for not ensuring a physical match between renewable energy generation and consumption \cite{spglobal-recs, bock-icelandGOproblem, re100report-2020}, and for absence of evidence that such schemes drive build-out of additional renewable energy capacity \cite{bjorn-RECSnatcom-2022, gillenwater-2014}.

Some corporate and industrial (\gls{ci}) buyers recongnised the problems associated with the unbundled certificates and turned towards Power Purchase Agreements (\gls{ppa}s). 
When signing a \gls{ppa}, energy buyers pledge to buy both the energy and the environmental credits \enquote{bundled} to megawatt-hours.
Market-driven procurement of renewable energy has become one of the major drivers of the energy transition: \gls{ppa}s are expected to account for one-fifth of utility solar PV and wind capacity expansion in 2023--2024, and almost twice as much (36\%) when China is excluded \cite{iea-REppa2023}.
In some countries, such as Spain and Sweden, market-driven procurement of renewable energy capacity surpasses policy-driven capacity expansion \cite{iea-REppa2023}.

\textbf{The key problem --} When \gls{ci} consumers purchase renewable credits, either via unbundled schemes or via bundled \gls{ppa}s, renewable energy supply is typically matched over a long period of time with buyers' energy demand.
As an example, more than 400 members of the RE100 group \cite{re100report-2020} have committed to purchasing renewable energy that matches 100\% of their electricity consumption on an annual basis.
Even the \enquote{best case} of 100\% annual matching commitments, where energy buyers sign \gls{ppa}s with renewable energy generators located within a local grid, has a major problem: the temporal mismatch between renewable energy supply and electricity demand.
In hours when wind and solar generation is low, energy buyers have to depend on carbon-emitting technologies, like coal and gas, available on local markets.
As a result, the 100\% annual renewable matching approach does not entirely eliminate carbon emissions from the energy buyer's portfolio.
Furthermore, the 100\% annual matching approach keeps electricity buyers exposed to price volatility, as the price of electricity on the wholesale markets can vary significantly from hour to hour.

% 3. **24/7 Carbon-free Energy Procurement**
% Introduce the 24/7 CFE idea
% Initiatives corporate and NGO interest in 24/7 Carbon-free Energy Procurement

\textbf{One proposed solution: 24/7 CFE --} There is growing interest from the leaders in the clean energy procurement to move beyond the 100\% annual matching approach.
One approach is called 24/7 Carbon-Free Energy (\gls{cfe}) procurement, which aims to match clean energy supply with electricity demand on an \textit{hourly basis}.
Thus, 24/7~CFE procurement has the potential to eliminate all greenhouse gas emissions associated with electricity use of the participating companies.
Furthermore, 24/7~CFE procurement is focused on \textit{carbon-free} rather than renewable energy, which opens up opportunities for a broad range of advanced energy technologies, such as long-duration energy storage (\gls{ldes}) or new clean firm power generators that are needed for deep decarbonisation of energy systems.

Google led the 24/7 movement in 2020 by announcing its commitment to cover the demand at all its data centres and campuses worldwide using the 24/7 CFE approach by 2030 \cite{google-247by2030}.
Shortly after, Google published a policy roadmap on achieving the goal \cite{google-PolicyRoadmap}.
Also, Microsoft has committed to the 100/100/0 by 2030 goal (i.e. \enquote{100\% of electricity consumption, 100\% of the time, matched by zero carbon energy purchases} \cite{Microsoft-vision}. 
The list of entities that have committed to 24/7 CFE procurement includes also utilities, such as Peninsula Clean Energy (a community choice aggregator in California) committed to 24/7~CFE by 2025 \cite{peninsula-OurPathto247}, and even cities like Des Moines, Iowa U.S., with a commitment by 2035 \cite{iowaenvcouncil-247}. In 2021, an international group of energy suppliers, electricity buyers, and \gls{ngo}s launched the 24/7 Carbon-free Energy Compact \cite{gocarbonfree247}.
With over 135 signatories, the group aims to \enquote{develop and scale high-impact technologies, energy policies, procurement practices, aand solutions to make 24/7 Carbon-Free Energy achievable for all}.

% 5. **24/7 model-based studies review**

\textbf{Existing literature on 24/7 CFE --} The growing interest in the means, costs and the system-level impacts of hourly \gls{cfe} procurement has created a need for a quantitative analysis of the approach.
There is relatively little research on the system-level impacts of 24/7~CFE procurement; and it is all fairly recent.
The first to incorporate it into the electricity system planning model were Xu et al. (2022) \cite{xu-247CFE-SSRN}.\footnote{A working paper updates a prior report published in November 2021 investigating the system-level impacts of 24/7~CFE in two regions of the U.S. -- California and PJM \cite{xu-247CFE-report}.} 
The authors study the system-level impacts of 24/7~CFE procurement based on an example of California in the year 2030.
In November 2022, IEA also released a model-based study on the 24/7~CFE, focusing on India and Indonesia \cite{ieaAdvancingDecarbonisationClean2022}.
In early 2023, Peninsula Clean Energy published an own white paper introducing their modeling tool---MATCH (Matching Around-the-Clock Hourly energy)---and the results of the modeling indicating feasibility of the company's \enquote{24/7 renewable energy by 2025} goal \cite{peninsula-report247}.

The first studies indicate that voluntary 24/7~CFE commitments reduce the emissions of participating buyers as well as emissions in the electricity grids they operate.
What has been missing is a comprehensive analysis of 24/7~CFE procurement over multiple regions and years, since electricity system characteristics, National Climate and Energy Plans (\gls{necp}s), renewable resource quality, and the cost of clean energy vary significantly across regions and over time.
Furthermore, it is unclear whether decarbonisation effect of 24/7 CFE procurement remains relevant when electricity grids are progressively decarbonised, as is the case in many countries and jurisdictions, or if the 24/7~CFE procurement is only impactful in the early stages of the energy transition.

% 6. **Relevance of this paper**

\textbf{Contribution --} The novelty of this study is to go beyond case studies and show in a systematic way, across many regions and levels of energy transition, how voluntary commitments to 24/7 CFE procurement impact participating buyers and regions where voluntary procurement occurs.
We explore different designs, optimal procurement strategies, costs, benefits and system-level impacts of the 24/7~CFE matching approach.
Our analysis focuses on four regions within the European energy system with unique characteristics, and we vary the analysis years to track the evolution of the energy system.
Additionally, we examine the underlying mechanisms that drive the system-level reduction of \co emissions, and examine how they perform over time and across regions. 
This work is a follow-up to the earlier public report by the authors \cite{riepin-zenodo-systemlevel247} with an updated model, updated system-wide assumptions, and additional analysis of the results. 

% 7. **Prologue**

The remainder of the paper is structured as follows: \cref{sec:methods} gives a brief summary of methodology, sources of model input data, and the experimental setup. 
The results are presented and discussed in \cref{sec:results} and put in perspective in \cref{sec:discussion}. 
The work is concluded in \cref{sec:conclusion}. 
\nameref{sec:si} provides further details about model assumptions, background energy system parametrisation, and the clean energy procurement mathematical model.