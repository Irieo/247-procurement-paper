\textbf{Sketch introduction - to refine}

Climate change is driving a global effort to rapidly decarbonise electricity systems across the globe. 
Many public and private energy buyers have joined this effort and highlight their sustainable credentials by procuring renewable energy with the Power Purchase Agreements (\gls{ppa}s).
\gls{ppa}s typically allow for renewable energy to match supply and demand on average over a long period of time. 
For example, more than 360 members of the there100.org group\footnote{\url{https://www.there100.org/}} have committed to purchasing enough renewable energy to match 100\% of their electricity consumption on an annual basis. 
Such voluntary commitments accelerate the deployment of wind and solar capacity above the policy requirements in the countries and jurisdictions these companies operate.

However, the energy buyers that commit to 100\% annual matching from Renewable Energy Sources (\gls{res}) still face times when generation from wind and solar generators is not sufficient to match the companies' electricity demand.
During those hours, the energy buyers often have to rely on carbon-emitting power generating technologies available on the local market, such as coal and gas. 
The energy demand supplied by the annual matching approach is thus not \textit{entirely} carbon-free. 
Furthermore, 100\% annual matching approach keeps electricity buyers exposed to price volatility. 
Finally, it requires the energy system to maintain backup and flexibility options for hours with low renewable generation.

For this reason, there is growing interest from corporations such as Google to match their demand with clean energy supply on a truly 24/7 basis. 
Achieving 24/7 Carbon-Free Energy (CFE) means that every kilowatt-hour of electricity consumption is met with carbon-free electricity sources, every hour of every day. 
Thus, this approach has the potential to eliminate all carbon emissions associated with Google's electricity use completely. 
In 2020 Google committed to the ambitious goal of operating entirely on a 24/7-CFE approach at all its data centres and campuses worldwide by 2030 \cite{Google-1}. 
Shortly after, Google published a policy roadmap on achieving the 24/7-CFE goal \cite{Google-2}.

In this study, we model 24/7-CFE procurement in Europe up to 2030. 
We explore different designs, costs, benefits and impacts of the 24/7 approach, both for Google and for the power systems in which the data centres are located. 
In sum, we want to find out: 

\begin{enumerate}
    \item How can we achieve hourly clean energy matching? 
    \item What is the 24/7-CFE cost premium versus the 100\% annual matching with renewables? 
    \item To which extent can technologies, such as long-duration storage or new dispatchable clean technologies, help to achieve the 24/7-CFE goal? 
    \item If many companies follow the 24/7-CFE approach, how does this effect the rest of the energy system?
\end{enumerate}