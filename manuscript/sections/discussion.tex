\textbf{Our results in perspective --} Our findings agree with the existing model-based studies on clean energy procurement \cite{xu-247CFE-SSRN,ieaAdvancingDecarbonisationClean2022,peninsula-report247}, as well as with the general understanding of the energy research community that \enquote{100\% renewable energy is not enough} \cite{chalendar-2019} in a context of electricity consumers claiming environmental sustainability. 
Specifically, we show that through hourly matching of demand with clean electricity, electricity consumers can both completely negate their carbon emissions and contribute to broader system-wide decarbonisation. 
100\% CFE matching is viable with commercially available technologies, albeit incurs a cost premium for participating consumers.
The cost premium can be substantially reduced if the advanced technologies, such as long-duration energy storage and clean firm generators are available, or if consumers opt for participating in the 24/7~CFE procurement with quality scores ranging from 90\% to 95\% instead of an absolute 100\%.

Another important finding of this work that confirms prevailing understanding is that 24/7 CFE matching creates an early market for advanced energy technologies. Citing an episode of the David Roberts' Volts podcast: \enquote{One reason energy nerds are excited about the 24/7 trend is that it's going to pull forward in time a bunch of questions (and investment decisions) that were going to face grids trying to reach 100 percent CFE anyway} \cite{roberts-intro247CFE}. Further research is needed to better understand how 24/7 CFE commitments push advanced energy technologies along their learning curves, and to estimate the potential economic impact on global decarbonisation costs.

% Where we add: Diverse grid locations and years
According to a comparison of the results over the modeled regions and time periods, voluntary clean energy procurement commitmetns appear to have consistent effects on both participating buyers and system-level decarbonisation.
Based on these findings, this work can provide a basis for generalising the implications of clean energy procurement strategies, while taking into account regional differences in electricity demand, renewable resource availability, legacy power plant fleets, degrees of interconnection, and energy policies, among other factors.

% Where we add: Decarbonisation story
A deeper look at the role of voluntary clean energy procurement commitments in grid decarbonisation reveals that the hourly matching strategy retains its significance in reducing system-level emissions, even as grids become cleaner over time.
We examined two distinct mechanisms through which 24/7 CFE procurement facilitates system decarbonization: "profile" and "volume".
As companies adapt their procurement strategies, it is important to understand how these mechanisms perform across different grid locations and years.
Interestingly, the disparity between the annual renewable (\enquote{volumentric}) matching and 24/7 CFE hourly matching strategies in decarbonisation outcomes becomes ever more evident as electricity grids become cleaner.
This phenomenon illustrates the effectiveness of 24/7 CFE procurement in supporting system decarbonisation within the context of evolving national energy and climate policies.


\textbf{Features of 24/7~CFE not covered in this work --} Some positive features of 24/7 CFE procurement fall beyond the scope of this manuscript.

% on the impact on back up flexibility needs
From a system-wide benefits perspective, voluntary commitments to 24/7 CFE reduce the need for flexibility in background electricity systems. 
There will be less investment in gas-fired peaker generators, as well as less need for battery storage elsewhere in the system.
This effect has been explored in Xu et al. (2022) and Riepin~\&~Brown (2022) \cite{riepin-zenodo-systemlevel247,xu-247CFE-report}

% add a point on the price hedge
From an electricity buyer perspective, 24/7 CFE procurement can provide a hedge against volatile wholesale market prices.
Since 100\% CFE matching implies covering the entire electricity demand with clean energy from procured generators via power purchase agreements and operating storage, the buyer is not exposed to the wholesale market price risk (see \cref{fig:10-2025-IE-p1-used}).
More information on the price hedge effect is provided by the 24/7 CFE Energy Compact \cite{gocarbonfree247}.


Diverse grid locations:
The results highlight that the decarbonisation impact of 24/7 hourly procurement can be achieved with very little cost premium over the 100\% annual renewable matching, if the advanced technology is available.
