\textbf{Our results in perspective --} Our findings agree with the existing model-based studies on clean energy procurement \cite{xu-247CFE-SSRN,ieaAdvancingDecarbonisationClean2022,peninsula-report247}, as well as with the general understanding of the energy research community that \enquote{100\% renewable energy is not enough} \cite{chalendar-2019} in a context of electricity consumers claiming environmental sustainability. 
Specifically, we show that through hourly matching of demand with clean electricity, electricity consumers can both completely negate their carbon emissions and contribute to broader system-wide decarbonisation. 
100\% CFE matching is viable with commercially available technologies, albeit incurs a cost premium for participating consumers.
The cost premium can be substantially reduced if the advanced technologies, such as long-duration energy storage and clean firm generators are available, or if consumers opt for participating in the 24/7~CFE procurement with quality scores ranging from 90\% to 95\% instead of an absolute 100\%.

% Where we converge: on advanced technologies
Another important finding of this work that confirms prevailing understanding is that 24/7 CFE matching creates an early market for advanced energy technologies. Citing an episode of the David Roberts' Volts podcast: \enquote{One reason energy nerds are excited about the 24/7 trend is that it's going to pull forward in time a bunch of questions (and investment decisions) that were going to face grids trying to reach 100 percent CFE anyway} \cite{roberts-intro247CFE}. Further research is needed to better understand how 24/7 CFE commitments push advanced energy technologies along their learning curves, and to estimate the potential economic impact on global decarbonisation costs.

% Where we add: Diverse grid locations and years
According to a comparison of the results over the modeled regions and time periods, voluntary clean energy procurement commitments appear to have consistent effects on both participating buyers and system-level decarbonisation.
Based on these findings, this work can provide a basis for generalising the implications of clean energy procurement strategies, while taking into account regional differences in electricity demand, renewable resource availability, legacy power plant fleets, degrees of interconnection, and energy policies, among other factors.

% Where we add: Decarbonisation story
A deeper look at the role of voluntary clean energy procurement commitments in grid decarbonisation reveals that the hourly matching strategy retains its significance in reducing system-level emissions, even as grids become cleaner over time.
We examined two distinct mechanisms through which 24/7 CFE procurement facilitates system decarbonization: "profile" and "volume".
As companies adapt their procurement strategies, it is important to understand how these mechanisms perform across different grid locations and years.
Interestingly, the disparity between the annual renewable (\enquote{volumentric}) matching and 24/7 CFE hourly matching strategies in decarbonisation outcomes becomes ever more evident as electricity grids become cleaner.
This phenomenon illustrates the effectiveness of 24/7 CFE procurement in supporting system decarbonisation within the context of evolving national energy and climate policies.


\textbf{Features of 24/7~CFE not covered in this work --} Some positive features of 24/7 CFE procurement fall beyond the scope of this manuscript.

% on the impact on back up flexibility needs
From a system-wide benefits perspective, voluntary commitments to 24/7 CFE reduce the need for flexibility in background electricity systems. 
There will be less investment in gas-fired peaker generators, as well as less need for battery storage elsewhere in the system.
This effect has been explored by Xu et al. (2022) and Riepin~\&~Brown (2022) \cite{riepin-zenodo-systemlevel247,xu-247CFE-report}

% add a point on the price hedge
From an electricity buyer perspective, 24/7 CFE procurement can provide a hedge against volatile wholesale market prices.
Since 100\% CFE matching implies covering the entire electricity demand with clean energy from procured generators via power purchase agreements and operating storage, the buyer is not exposed to the wholesale market price risk (see \cref{fig:10-2025-IE-p1-used}).
More information on the price hedge effect is provided by the 24/7 CFE Energy Compact \cite{gocarbonfree247}.


\textbf{Critical appraisal --} This work has simplifications and limitations that should be acknowledged.

%Limitations: electricity sector only
First, only the electricity sector is included in the mathematical model of the European energy system.
The research focus of this study justifies it.
However, it is important to acknowledge that the electricity sector is not an isolated system, but it is closely linked with other energy sectors, such as heat and transport.
Brown et al. (2018) discuss the synergies of sector coupling in the integrated European energy system \cite{brownSynergiesSectorCoupling2018}.

%Limitations: merged profiles of consumers
Second, consumers participating in clean energy procurement are modelled as one entity.
The real-world situation may involve a number of individual consumers participating in a voluntary clean energy procurement scheme.
Our assumption is that all consumers that commit to 24/7 matching, form alliances, and enter into contracts with carbon-free energy generators so that their aggregated consumption can be matched with clean generation on an hourly basis to achieve a given CFE target.
In reality, participating consumers can pursue hourly matching strategies independently based on their own load profiles.
See Xu \& Jenkins (2022) investigating this case \cite{princeton-TEACs-2022}.

% Limitations: consumption profiles
Third, we assume that the participating consumers have a flat demand profile (i.e. baseload).
As a matter of fact, consumer demand profiles vary based on their business activities.
It has been shown by Riepin~\&~Brown (2022) that the shape of consumption profiles affects the cost-optimal technology mix required for achieving a certain CFE target; however, the shape of the profile has relatively little impact on procurement costs, emissionality of the portfolio, and system-level impacts of 24/7 procurement \cite{riepin-zenodo-systemlevel247}.

% Limitations: a point on DSM
Fourth, participating consumers have an inflexible demand, i.e., no demand-side management is considered.
In reality, many \gls{ci} consumers have some degree of demand flexibility, which they can use to adjust their consumption as needed based on grid signals, such as the wholesale market price and the average emission rate of the local electricity mix.
Moreover, some consumers, such as data centers and cloud computing providers, can shift their load also in space, i.e., to shift computing
jobs and associated power loads across data center locations \cite{rosskoningsteinWeNowMore2021}.
In the recent work, Riepin~\&~Brown (2023) explore how space-time load-shifting flexibility can be used to meet high 24/7 CFE targets, as well as what potential benefits it may offer to participating energy buyers and to the rest of the energy system \cite{riepin-zenodo-spacetime247CFE}.

% Excess constraint problem
One of the simplifications of mathematical models implementing 24/7 CFE procurement is how the excess carbon-free electricity is handled. 
The \enquote{excess CFE} can be defined as the total amount of carbon-free electricity supply exceeding the demand of the participating consumers in a given period of time.
Here we assume that the excess can either be curtailed or sold to the regional electricity market at wholesale market prices, until the excess constraint is reached (see \cref{sec:si_3} for detail).
The problem with the uncostrained export of excess CFE to the regional market could be a situation where excess electricity from the renewable generators procured by 24/7 consumers displaces electricity from other renewable generators in the market, which in turn may challenge additionality of the 24/7 procurement. 
Similar \enquote{spillover effect} problem has been observed and discussed by Xu et al. (2022) \cite{xu-247CFE-SSRN}.
In their white paper, Peninsula Clean Energy team also discuss the problems and implications of the excess supply. Citing their work: \enquote{We conclude that excess supply is a necessary aspect of a time-coincident renewable portfolio in the range of likely market conditions in California today} \cite{peninsula-report247}.
Both studies perform sensitivity analyses on the excess supply and discuss the implications on their findings \cite{xu-247CFE-SSRN, peninsula-report247}.

% On "emissionality" approach
\textbf{Is 24/7~CFE the right strategy? --} Last but not least, it is worth mentioning an ongoing discussion whether 24/7 CFE is the right strategy for companies to reduce their carbon emissions.
Some proponents of alternative strategies contend that 24/7 CFE, or \enquote{temporal matching}, may not represent the most effective approach to emissions reduction. 
Instead, they suggest that companies, in their pursuit to reduce system-level carbon emissions, should focus on system-wide impact as measured via short-run marginal emissions accounting (\enquote{emissions matching}).
This argument is based on the rationale that optimising procurement and deployment of carbon-free generators and batteries to cover 24/7 consumption will be different from optimising the assets to \enquote{offset emissions} based on the short-run marginal emission estimates of the grid.
Xu~et~al. (2023) bring this discussion from the realm of arguments into quantitative analysis, discovering that \enquote{emissions matching strategies have zero or near-zero long-run impact on system-level \co emissions} \cite{princeton-247CFEvsEmissionality}.

More work---encompassing the collective contributions of the research community, \gls{ngo}s, policy-makers, and corporate energy buyers---is needed to deepen our understanding of the implications, challenges, and opportunities inherent in various clean energy procurement strategies, guiding us all towards net-zero sustainable energy systems.