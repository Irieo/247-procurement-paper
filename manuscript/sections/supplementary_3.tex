%S3: Mathematical formulation of clean energy procurement
We introduce new model components (parameters, variables, constraints) into the PyPSA-Eur model to represent voluntary clean energy procurement.
By incorporating these into the mathematical problem, we model a situation where a fraction of electricity demand in a particular bidding zone is voluntarily committed to procuring clean energy with a desired strategy.
The fraction is hereinafter referred to as \enquote{participating consumers}.

\textbf{100\% annual matching --} The annual (\enquote{volumetric}) renewable matching strategy is modelled with a constraint (\ref{eqn:RES100}). 
The constraint requires participating consumers to purchase just enough renewable electricity to match all of their electricity consumption on an annual basis.

More formally, the sum of all dispatch $g_{r,t}$ of contracted renewable generators $r\in RES$ over the year $t\in T$ is equal to the sum of the annual demand $d_t$ of participating consumers:
\begin{equation}
\sum_{r\in RES, t\in T} g_{r,t} = \sum_{t\in T} d_t
\label{eqn:RES100}
\end{equation}

The contracted renewable generators must be new, i.e., they must be additional to the system.
Procured generators can be sited only in the local bidding zone; thus, we model the \enquote{best case} of the annual matching strategy.

\textbf{24/7 CFE hourly matching --} The hourly matching strategy is modelled with a constraint (\ref{eqn:CFE}). In this case, participating consumers commit to match their electricity demand with carbon-free electricity (CFE) throughout the year on an hourly basis. The 24/7~CFE procurement framework is based on the methodologies and metrics paper by Google~(2021) \cite{google-methodologies}.

More formally, the hourly generation from the procured CFE generators $r\in CFE$, discharge and charge from the procured storage technologies $s\in STO$, plus imports of electricity from the regional grid $im_t$ multiplied by the grid's hourly CFE factor $CFE_t$ minus the excess of the CFE supply must be higher or equal than a certain CFE target $x$ multiplied by the total load of participating consumers:

\begin{equation}
    \begin{split}
        \sum_{r\in CFE, t\in T} g_{r,t} &+ \sum_{s\in STO, t\in T} \left(\bar{g}_{s,t} - \underline{g}_{s,t}\right) \\
        &- \sum_{t\in T} ex_t + \sum_{t\in T} CFE_t \cdot im_t \geq x \cdot \sum_{t\in T} d_t
    \end{split}
\label{eqn:CFE}
\end{equation}

% On CFE target
\noindent where \textit{CFE target} $x$[\%] measures the degree to which hourly electricity consumption is matched with carbon-free electricity generation. 
Equation (\ref{eqn:CFE}) thus allows for controlling \textit{the quality score} of the 24/7~CFE procurement by adjusting the parameter $x$. The best quality score---100\% CFE---means that every kilowatt-hour of electricity consumption is met by carbon-free sources at all times.

% On locational matching and additionality
Similarly to the hourly matching strategy, the contracted generators must be additional to the system and can be sited only in the local bidding zone.

% On excess
Note that if total electricity generation of assets procured by participating consumers exceeds demand in a given hour, the \enquote{excess} carbon-free electricity is not counted toward a CFE target.
Here we assume that the excess can either be curtailed or sold to the regional electricity market at wholesale market prices.
We set a constraint on the total amount of excess generation sold to the regional grid, setting the limit to 20\% of the total annual demand of participating consumers.
The necessity of this limit is addressed in the \cref{sec:discussion}. The wholesale market prices are derived from dual variables of a nodal energy balance constraint. 

% On bilinear term due to grid CFE factor
The grid CFE factor $CFE_t$ can be seen as the percentage of clean electricity in each MWh of imported electricity to supply demand of participating consumers in a given hour.
Note that in equation (\ref{eqn:CFE}), $CFE_t$ is affected by additional CFE resources procured by the participating consumers. 
This introduces a nonconvex term to the optimisation problem. 
The nonconvexity can be avoided by treating the grid CFE factor as a parameter that is iteratively updated (starting with $CFE_t =0 \,~\forall t$).
We find that one forward pass (i.e. 2 iterations) yields very good convergence.

The methodology to calculate the grid hourly CFE factor is described in the prior work of the authors \cite{riepin-zenodo-systemlevel247}.